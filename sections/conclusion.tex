In conclusion, our discussion highlights that the choice between graph and relational databases should be guided by the specific requirements of the application, the nature of the data, and the desired performance characteristics. Future research could explore hybrid models that leverage the strengths of both graph and relational databases, potentially providing a more flexible and efficient solution for a broader range of applications. Additionally, advancements in graph database technologies and query optimization techniques may further enhance their performance and applicability in managing large-scale, interconnected data.

% What we have done:
% demonstrated the difference with the relational database model. / the paradigm differences between graph databases and relational databases. 
% we demonstrated the differences between the query languages used in graph databases(Cypher) and relational databases(SQL).
% we compared the differences in how they store the data.
% we looked at other papers on graph databases and relational databases, comparing them, looking at previous work, and seeing what they have done.
%       we found that performance 
%       we found that they are more bendable


% What has been done in this paper
% What are the results
% What are the implications of the results
% What are the future directions
% What are the limitations of the study

