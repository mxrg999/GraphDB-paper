\subsection{An example in Cypher and SQL}
In order to highlight the potential that graph databases offers, we will in this section look at a few examples queries in SQL (Structured Query Language) and in Cypher. 
As mention above, Neo4j does not use standard SQL for interacting with the database. Instead it uses a query-language called Cypher. Cypher is a declarative language that has a similar syntax to SQL. Because Cypher is used when interacting with graph databases it offers some functionality not found in SQL. The syntax in Cypher is visual in nature as a node in Cypher is expressed within parentheses and a relation is expressed within square brackets. A query containing both nodes and relationships can therefore be seen as a small graph with labeled edges. Let us look at some examples in Cypher and compare them with the corresponding SQL. The following Cypher examples are from the official documentation for Neo4j. \footnote{Neo4j docs - Basic queries 2024-05-20 \url{https://neo4j.com/docs/cypher-manual/current/queries/basic/}} In these examples we assume that we already have a database that contains information regarding actors and movies. Here is an example query in Cypher:
\begin{sqlCode}
  MATCH (keanu:Person {name:'Keanu Reeves'})
  RETURN keanu.name AS name, keanu.born AS born
\end{sqlCode}
Here we query the database after a node that has the property name 'Keanu Reeves', if we find such a node we return a table with that nodes \texttt{name} and \texttt{born} properties. The table would look like this:
\begin{center}
  \begin{tabular}{|c|c|}
    \hline
    name           & born \\
    \hline
    "Keanu Reeves" & 1964 \\
    \hline
  \end{tabular}\\
\end{center}
Here is a SQL query that would return a similar table from a relational database:


\begin{sqlCode}
SELECT
  name AS name,
  born AS born
FROM
  Person
WHERE
  name = 'Keanu Reeves';
\end{sqlCode}

For simple queries as the ones above there are little to no difference between Cypher and SQL. However some kind of queries are better suited for graph databases. Such queries are for example when the information that we want to find out is not necessarily structured inside the database. Let us look at an example, suppose that we want to find out if there is any direct or indirect relation between two actors, such a relation could be that they both stared in a movie together, or that they both have worked with the same director in different movies or a even more distant relation. This query returns the shortest path if such a relation exist:

\begin{sqlCode}
MATCH p=shortestPath(
(:Person {name:"Keanu Reeves"})-[*]-(:Person{name:"Tom Hanks"})
) RETURN p
\end{sqlCode}

The query above can be read as; find the shortest path between the node with the name property "Keanu Reeves" and the node with the name property "Tom Hanks". The inner query returns all relation of any length between the two nodes and the function \texttt{shortestPath()} returns the shortest relation between them. The \texttt{-[*]-} between the two nodes matches all direct and indirect relations between them. An similar query in SQL could be constructed the following way:
\begin{sqlCode}
WITH RECURSIVE ShortestPath AS (
SELECT
  id,
  name,
  ARRAY[id] AS path,
  0 AS depth
FROM
  people
WHERE
  name = 'Keanu Reeves'

UNION ALL

SELECT
  p.id,
  p.name,
  sp.path || p.id,
  sp.depth + 1
FROM
  ShortestPath sp
JOIN
  relationships r ON sp.id = r.person1_id
JOIN
  people p ON r.person2_id = p.id
WHERE
  NOT p.id = ANY(sp.path)
  )
SELECT *
FROM ShortestPath
WHERE name = 'Tom Hanks'
ORDER BY depth
LIMIT 1;

\end{sqlCode}
In this query we have to use a recursive common table expression in order to achieve the same functionality as in the Cypher query. This example shows that some functionality in relational database requires more complexity from the query language. The difference in length between the two queries shows the difference in complexity. Queries of this nature is where graph databases are more appropriate to use rather than relational databases. The specifics of the database that is being queried and how much data that is in said database does impact the efficiency of the queries, but less complexity is often preferable when interacting with databases. Graph databases has the potential to be as fast or faster than relational databases. \footnote{López, F.M.S., De La Cruz E. G. S. (2015) Literature review about Neo4j graph database as a feasible alternative for replacing RDBMS. Network of Scientific Journals from Latin America, the Caribbean, Spain and Portugal} However in order to conclude any actual performance difference the datasets have to be much bigger then the dataset used in the examples above.



