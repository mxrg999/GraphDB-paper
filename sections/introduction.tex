% Text example: Graph databases have garnered attention for their efficiency in managing complex and interconnected data. Unlike traditional relational databases, graph databases are designed to handle the intricacies of large-scale connections. This paper evaluates graph databases in comparison to relational databases, focusing on their application in real-world scenarios such as social networks and recommendation systems. We outline our methodology, datasets, and the structure of our comparative analysis.

% Introducing the concept of graph databases, 

\subsection{Introduction}


Its approach of storage of data through an arrangement of nodes, edges and relationships
index-free adjacency. 

stored is associative and not indexed.


Graph databases are very automatic and unrestrained method of describing any kind of data. 

Graph databases have gained popularity in recent years due to their ability to efficiently manage complex and interconnected data structures. Unlike traditional relational databases, 

% Real-world application examples where graph databases shine 
%   (e.g., social networks, recommendation systems).

Compare it to relational databases, tables vs graphs, show an image of a table and a graph


A graph database is modestly a database which is grounded on a graph data structure. 
Similar to a graph, it can accumulate nodes and edges amongst nodes. 

An edge having label defines association amongst two nodes

Graph databases are one of the four classifications of NoSQL databases. 
The other classifications of NoSQL databases are: key-value, documents and column oriented.






% Stating the purpose of our paper: 
%   which is to compare graph databases with traditional relational databases, and to show when graph databases are a useful technology.



% An outline of the overall structure.
