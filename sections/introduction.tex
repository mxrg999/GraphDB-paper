The landscape of database technologies has evolved significantly over the past few decades. Traditionally, relational databases, based on the relational model introduced by E.F. Codd\footnote{E.F. Codd, "A Relational Model of Data for Large Shared Data Banks," \textit{Communications of the ACM}, vol. 13, no. 6, pp. 377-387, 1970. Available: \url{https://cs.uwaterloo.ca/~david/cs848s14/codd-relational.pdf}} in the 1970s, have been the cornerstone of data storage and management. These databases organize data into tables with predefined schemas, making them suitable for structured data and transactional applications. However, during the 2000s, the rise of web-scale applications, social networks, and big data exposed the limitations of relational databases in handling highly interconnected and dynamic data.

Graph databases have emerged as a powerful alternative to relational databases for managing complex and interconnected datasets. Unlike relational databases, graph databases model data as nodes (entities) and edges (relationships), allowing for more natural and flexible data representation. This approach is particularly effective for applications that require traversing relationships, such as social networks, recommendation systems, and fraud detection.

Despite the growing interest in graph databases, there is still a need for a comprehensive comparison with traditional relational databases to understand their respective strengths and limitations. Developers and data scientists often face challenges in selecting the appropriate database technology for their specific needs, especially when dealing with complex data structures and relationships.

The primary objective of this paper is to provide a comparative study between graph databases and relational databases. Using Neo4j as a representative of graph databases and PostgreSQL as a representative of relational databases, we aim to evaluate their performance, efficiency, and suitability for different types of applications.


This paper is structured as follows:
\begin{itemize}
    \item \textbf{Section 2:} An overview of graph databases, including their types and unique characteristics.
    \item \textbf{Section 3:} A detailed comparison between graph databases and relational databases, focusing on paradigm differences, query languages, and data modeling.
    \item \textbf{Section 4:} A discussion of related work, highlighting existing research and studies on this topic.
    \item \textbf{Section 5:} A summary of the key findings from the comparative study.
    \item \textbf{Section 6:} Our conclusions and suggestions for future work.
\end{itemize}
