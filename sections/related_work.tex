% Finding other papers on graph databases and relational databases, comparing them, looking at previous work, and seeing what they have done. 

The debate between relational databases and NoSQL databases, particularly graph databases, has been extensively studied. In their comprehensive review, Douglas Kunda and Hazael Phiri compare relational and NoSQL databases, including graph databases. They state, "Relational Databases have poor scalability, weak performance, cost more, face availability challenges when supporting a large number of users, and handle limited volume of data. NoSQL, on the other hand, is based on the BASE model, which emphasizes greater scalability and provides a flexible schema, offers better performance, is mostly open source, cheap but lacks a standard query language and does not provide adequate security mechanisms." They conclude that "both databases will continue to exist alongside each other with none being better than the other."\footnote{Kunda, D., \& Phiri, H. (2017). A Comparative Study of NoSQL and Relational Database. Zambia ICT Journal, 1(1), 1-4. \url{https://doi.org/10.33260/zictjournal.v1i1.8}}.

The need for scalable and efficient data storage solutions has grown with the rise of web-scale applications, mobile technologies, and social media, leading to the advent of various NoSQL databases. According to W. Puangsaijai and S. Puntheeranurak, big data applications necessitate technologies that can efficiently handle large volumes of unstructured data. Their study compares Redis, a key-value NoSQL database, with MariaDB, a popular relational database, under various conditions. They find that "Redis has better runtime performance for insert, delete, update transactions under specific conditions or complex queries," whereas "MariaDB performs well with smaller datasets."\footnote{Puangsaijai, W., \& Puntheeranurak, S. (2017). A comparative study of relational database and key-value database for big data applications. 2017 International Electrical Engineering Congress (iEECON), Pattaya, Thailand, 2017, pp. 1-4. \url{https://doi.org/10.1109/IEECON.2017.8075813}}.

Relational databases, though efficient for traditional data-intensive applications, struggle with complex queries involving many relationships. Chad Vicknair et al. compare Neo4j, a graph database, with MySQL, a relational database, for managing data provenance. They report that "relational database systems are generally efficient unless the data contains many relationships requiring joins of large tables." Their study demonstrates the efficiency of Neo4j for data provenance due to its inherent handling of relationships.\footnote{Vicknair, C., Macias, M., Zhao, Z., Nan, X., Chen, Y., \& Wilkins, D. (2010). A comparison of a graph database and a relational database: a data provenance perspective. Proceedings of the 48th Annual Southeast Regional Conference. \url{https://doi.org/10.1145/1900008.1900067}}.

Graph databases, such as Neo4j, have gained popularity for managing connected data. José Guia and colleagues evaluate Neo4j's performance and find that it excels in handling connected data but faces challenges with large data loads. They note, "Neo4j stands out for its simplicity and robust performance in loading and querying graph data, although it struggles with very large datasets."\footnote{Guia, J., Soares, V. G., \& Bernardino, J. (2018). Graph Databases: Neo4j Analysis. ISEC, Polytechnic of Coimbra; Informatics Centre, Federal University of Paraiba; CISUC - Centre for Informatics and Systems of the University of Coimbra}.

Finally, the growing use of geospatial data in applications like smart cities and disaster management underscores the need for efficient data management systems. M. Sharma et al. compare PostgreSQL, MongoDB, and Neo4j for managing geotagged data. Their analysis shows that "while RDBMS like PostgreSQL is effective, NoSQL databases like MongoDB and Neo4j offer better performance for handling voluminous, heterogeneous geospatial data."\footnote{Sharma, M., Sharma, V. D., \& Bundele, M. M. (2018). Performance Analysis of RDBMS and NoSQL Databases: PostgreSQL, MongoDB and Neo4j. 2018 3rd International Conference and Workshops on Recent Advances and Innovations in Engineering (ICRAIE), Jaipur, India. \url{https://doi.org/10.1109/ICRAIE.2018.8710439}}.

These studies collectively highlight the evolving landscape of database technologies, emphasizing that while relational databases continue to be valuable, NoSQL databases, including graph databases, offer significant advantages for specific use cases involving large volumes of connected data.


% Numerous studies have investigated the comparative efficiency of graph and relational databases.

% In their comprehensive review, Douglas Kunda and Hazael Phiri\footnote{Kunda, D., \\& Phiri, H. (2017). A Comparative Study of NoSQL and Relational Database. Zambia ICT Journal, 1(1), 1-4. https://doi.org/10.33260/zictjournal.v1i1.8} compare relational and NoSQL databases, including graph databases. They write, "Relational Databases have poor scalability, weak performance, cost more, face availability challenges when supporting large number of users and handle limited volume of data. NoSQL, on the other hand, is based on the BASE model, which emphasizes greater scalability and provides a flexible schema, offers better performance, mostly open source, cheap but, lacks a standard query language and does not provide adequate security mechanisms."

% They also state, "Both databases will continue to exist alongside each other with none being better than the other."



% % Graph databases are considerably bendable than relational database since new associations can be effortlessly supplementary on graph databases without rearrangement of the database schema again.
% https://ieeexplore.ieee.org/abstract/document/8995006